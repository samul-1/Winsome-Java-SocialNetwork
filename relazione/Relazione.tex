\documentclass[a4paper,8pt]{article} % Prepara un documento per carta A4, con un font di dimensione 12pt

\usepackage[italian]{babel} % Adatta LaTeX alle convenzioni tipografiche italiane,
% e ridefinisce alcuni titoli in italiano, come "Capitolo" al posto di "Chapter",
% se il vostro documento è in italiano
% l'opzione  linguistica 'french' è necessaria per l'abilitazione della
% successiva istruzione <<\frenchspacing>> 
\usepackage[T1]{fontenc} % Riga da togliere se si compila con PDFLaTeX
\usepackage[utf8]{inputenc} % Consente l'uso caratteri accentati italiani
\usepackage{amsthm}
\usepackage[margin=1.2in]{geometry}
\usepackage{hyperref}


\usepackage{listings}
\usepackage{xcolor}

\definecolor{codegreen}{rgb}{0,0.6,0}
\definecolor{codegray}{rgb}{0.5,0.5,0.5}
\definecolor{codepurple}{rgb}{0.58,0,0.82}
\definecolor{backcolour}{rgb}{0.95,0.95,0.92}

\lstdefinestyle{mystyle}{
    backgroundcolor=\color{backcolour},   
    commentstyle=\color{codegreen},
    keywordstyle=\color{magenta},
    numberstyle=\tiny\color{codegray},
    stringstyle=\color{codepurple},
    basicstyle=\ttfamily\footnotesize,
    breakatwhitespace=false,         
    breaklines=true,                 
    captionpos=b,                    
    keepspaces=true,                 
    numbers=left,                    
    numbersep=5pt,                  
    showspaces=false,                
    showstringspaces=false,
    showtabs=false,                  
    tabsize=2,
}

\lstset{style=mystyle}


\def\code#1{\texttt{#1}}
\title{Progetto di laboratorio di reti} % \LaTeX è una macro che compone il logo "LaTeX"
% I commenti (introdotti da %) vengono ignorati

\author{Samuele Bonini}
\date{}
% in alternativa a \date il comando \today introduce la data di sistema.

\begin{document}
\maketitle % Genera il titolo sulle istruzioni  \title, \author e \date



\tableofcontents % Prepara l'indice generale
\begin{flushleft}

    \section{Architettura di massima del progetto} %
    \textbf{Suddivisione in layer.}\quad \\ (img)\\
    ... Questo documento è strutturato in maniera tale da illustrare, come prima cosa, le caratteristiche delle entità
    di dominio e quali sono le operazioni primitive a esse associate, per poi spostarsi progressivamente sui layer
    più esterni dell'architettura, dando un'idea di come le classi appartenenti allo strato della business logic sono
    implementate, chiarendo infine come viene realizzata, agli atti, la comunicazione tra il server e il client.\\
    \textbf{Protocollo applicazione.}\quad (rest, token authentication)\\


    \section{Le entità di dominio}
    \textbf{Gli utenti.}\quad \\
    \textbf{Post, commenti, reazioni.}\quad \\
    \textbf{Classi richiesta e risposta.}\quad \\
    \textbf{Serializer.}\quad \\


    \section{Il data store}
    Introduzione\\
    \textbf{Strutture dati.}\quad \\
    \textbf{Persistenza dello stato.}\quad \\


    \section{Il service layer}
    Introduzione\\
    \textbf{La classe SocialNetworkService.}\quad \\
    \textbf{La classe RewardIssuer.}\quad \\
    \textbf{La classe WalletConversionService.}\quad \\
    \textbf{I servizi di registrazione utente e notifica.}\quad \\




    \section{L'API e il router}

    \textbf{La classe ApiRoute.}\quad \\
    \textbf{La classe ApiRouter.}\quad \\
    \textbf{Gestione di una richiesta HTTP.}\quad \\
    (img)\\

    \section{Il server}

    \textbf{Gestione delle connessioni.}\quad \\
    \textbf{La classe ServerConfig.}\quad \\

    \section{Il client CLI}

    \textbf{Gestione di una richiesta al server.}\quad \\
    \textbf{Rendering dei dati.}\quad \\
    \textbf{Gestione dei messaggi.}\quad \\

    \section{Il client GUI}

    \textbf{Realizzazione.}\quad \\
    \textbf{Limitazioni.}\quad \\


\end{flushleft}
\end{document}
